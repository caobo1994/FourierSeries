\documentclass[12pt, titlepage]{article}

\usepackage{booktabs}
\usepackage{tabularx}
\usepackage{hyperref}
\hypersetup{
    colorlinks,
    citecolor=black,
    filecolor=black,
    linkcolor=red,
    urlcolor=blue
}
\usepackage[round]{natbib}

%% Comments

\usepackage{color}

\newif\ifcomments\commentstrue

\ifcomments
\newcommand{\authornote}[3]{\textcolor{#1}{[#3 ---#2]}}
\newcommand{\todo}[1]{\textcolor{red}{[TODO: #1]}}
\else
\newcommand{\authornote}[3]{}
\newcommand{\todo}[1]{}
\fi

\newcommand{\wss}[1]{\authornote{blue}{SS}{#1}} 
\newcommand{\plt}[1]{\authornote{magenta}{TPLT}{#1}} %For explanation of the template
\newcommand{\an}[1]{\authornote{cyan}{Author}{#1}}

%% Common Parts

\newcommand{\progname}{FSL} % PUT YOUR PROGRAM NAME HERE %Every program
                                % should have a name


\begin{document}

\title{A Fourier Series Library: System Verification and Validation Plan for FSL} 
\author{Bo Cao}
\date{\today}
	
\maketitle

\pagenumbering{roman}

\section{Revision History}

\begin{tabularx}{\textwidth}{p{3cm}p{2cm}X}
\toprule {\bf Date} & {\bf Version} & {\bf Notes}\\
\midrule
Oct. 28, 2019  & 1.0 & First draft.\\
Date 2 & 1.1 & Notes\\
\bottomrule
\end{tabularx}

\newpage

\tableofcontents

\listoftables

\listoffigures

\newpage

\section{Symbols, Abbreviations and Acronyms}

Some symbols, abbreviations and acronyms are defined in the Common Analysis (CA) document \footnote{This document is available at \url{https://github.com/caobo1994/FourierSeries/blob/master/docs/SRS/CA.pdf}.}. For simplicity and maintainability, they are not redefined here. Readers shall refer to the CA documents when a certain item is not defined here.

\vspace{1cm}

\renewcommand{\arraystretch}{1.2}
\begin{tabular}{l l} 
  \toprule		
  \textbf{symbol} & \textbf{description}\\
  \midrule 
  T & Test\\
  \bottomrule
\end{tabular}\\

\newpage

\pagenumbering{arabic}

This document provides an overview of the Verification and Validation (VnV) plan for the Fourier Series Library (FSL). It lays out the purpose, methods, and test cases for the VnV procedure. 

\section{General Information}

\subsection{Summary}

The library to be tested is called the Fourier Series Library (FSL). This library performs a set of computations, transformations, and/or input/output at the request of the library user.

\subsection{Objectives}

The intended objective of the VnV procedure is to verify that this library has generally met the requirements described in the CA document. These requirements include the functional requirements (FRs) and the non-functional requirements (NFRs).

Note that if a small part of the NFRs has not been met, the library is still acceptable when the not-met NFRs' impact has been analyzed and deemed non-essential.

\subsection{Relevant Documentation}

As we said before, this document relies on the CA document. This document is also the base of the Unit Test Plan document.


\section{Plan}
	
\subsection{Verification and Validation Team}

The major member of the team is the author himself. Other contributors might assist in the VnV procedure, but their contributions are not guaranteed.

\subsection{CA Verification Plan}

The verification of the CA document mainly consists of the feedback from reviewers, and the author's experience in developing and verifying this library. 


\subsection{Design Verification Plan}

The design of this library will be verified by reviewing how the functions in this library relies on each other, and how they are integrated.

\subsection{Implementation Verification Plan}

The verification of the implementation of this library is mainly done by unit testing. The detail of unit testing can be found in the Unit Test Plan document. Mainly, the unit test will be done by first testing the basic functions in this library, and then testing the advanced functions. Please note that the test result of any function in this library is acceptable, if and only if the reliant functions of this function is tested to be right.
  


\subsection{Software Validation Plan}

The transformation part of this library will be validated by comparing its result with the MATLAB pseudo-oracle.

\section{System Test Description}
	
\subsection{Tests for Functional Requirements}

\wss{Subsets of the tests may be in related, so this section is divided into
  different areas.  If there are no identifiable subsets for the tests, this
  level of document structure can be removed.}

\wss{Include a blurb here to explain why the subsections below
  cover the requirements.  References to the SRS would be good.}

\subsubsection{Area of Testing1}

\wss{It would be nice to have a blurb here to explain why the subsections below
  cover the requirements.  References to the SRS would be good.  If a section
  covers tests for input constraints, you should reference the data constraints
  table in the SRS.}
		
\paragraph{Title for Test}

\begin{enumerate}

\item{test-id1\\}

Control: Manual versus Automatic
					
Initial State: 
					
Input: 
					
Output: \wss{The expected result for the given inputs}

Test Case Derivation: \wss{Justify the expected value given in the Output field}
					
How test will be performed: 
					
\item{test-id2\\}

Control: Manual versus Automatic
					
Initial State: 
					
Input: 
					
Output: \wss{The expected result for the given inputs}

Test Case Derivation: \wss{Justify the expected value given in the Output field}

How test will be performed: 

\end{enumerate}

\subsubsection{Area of Testing2}

...

\subsection{Tests for Nonfunctional Requirements}

\wss{The nonfunctional requirements for accuracy will likely just reference the
  appropriate functional tests from above.  The test cases should mention
  reporting the relative error for these tests.}

\wss{Tests related to usability could include conducting a usability test and
  survey.}

\subsubsection{Area of Testing1}
		
\paragraph{Title for Test}

\begin{enumerate}

\item{test-id1\\}

Type: 
					
Initial State: 
					
Input/Condition: 
					
Output/Result: 
					
How test will be performed: 
					
\item{test-id2\\}

Type: Functional, Dynamic, Manual, Static etc.
					
Initial State: 
					
Input: 
					
Output: 
					
How test will be performed: 

\end{enumerate}

\subsubsection{Area of Testing2}

...

\subsection{Traceability Between Test Cases and Requirements}

\wss{Provide a table that shows which test cases are supporting which
  requirements.}
				
\bibliographystyle{plainnat}

\bibliography{SRS}

\newpage

\section{Appendix}

This is where you can place additional information.

\subsection{Symbolic Parameters}

The definition of the test cases will call for SYMBOLIC\_CONSTANTS.
Their values are defined in this section for easy maintenance.

\subsection{Usability Survey Questions?}

\wss{This is a section that would be appropriate for some projects.}

\end{document}