\documentclass[12pt]{article}

\usepackage{amsmath, mathtools}
\usepackage{amsthm}
\usepackage{amsfonts}
\usepackage{amssymb}
\usepackage{graphicx}
\usepackage{colortbl}
\usepackage{xr}
\usepackage{hyperref}
\usepackage{longtable}
\usepackage{xfrac}
\usepackage{tabularx}
\usepackage{float}
\usepackage{siunitx}
\usepackage{booktabs}
\usepackage{caption}
\usepackage{pdflscape}
\usepackage{afterpage}

\usepackage[round]{natbib}

\usepackage{appendix}

%\usepackage{refcheck}

\hypersetup{
    bookmarks=true,         % show bookmarks bar?
      colorlinks=true,       % false: boxed links; true: colored links
    linkcolor=red,          % color of internal links (change box color with linkbordercolor)
    citecolor=green,        % color of links to bibliography
    filecolor=magenta,      % color of file links
    urlcolor=cyan           % color of external links
}

\input{../Comments}

% For easy change of table widths
\newcommand{\colZwidth}{1.0\textwidth}
\newcommand{\colAwidth}{0.13\textwidth}
\newcommand{\colBwidth}{0.82\textwidth}
\newcommand{\colCwidth}{0.1\textwidth}
\newcommand{\colDwidth}{0.05\textwidth}
\newcommand{\colEwidth}{0.8\textwidth}
\newcommand{\colFwidth}{0.17\textwidth}
\newcommand{\colGwidth}{0.5\textwidth}
\newcommand{\colHwidth}{0.28\textwidth}

% Used so that cross-references have a meaningful prefix
\newcounter{defnum} %Definition Number
\newcommand{\dthedefnum}{GD\thedefnum}
\newcommand{\dref}[1]{GD\ref{#1}}
\newcounter{datadefnum} %Datadefinition Number
\newcommand{\ddthedatadefnum}{DD\thedatadefnum}
\newcommand{\ddref}[1]{DD\ref{#1}}
\newcounter{theorynum} %Theory Number
\newcommand{\tthetheorynum}{T\thetheorynum}
\newcommand{\tref}[1]{T\ref{#1}}
\newcounter{tablenum} %Table Number
\newcommand{\tbthetablenum}{T\thetablenum}
\newcommand{\tbref}[1]{TB\ref{#1}}
\newcounter{assumpnum} %Assumption Number
\newcommand{\atheassumpnum}{A\theassumpnum}
\newcommand{\aref}[1]{A\ref{#1}}
\newcounter{calnum} %Calculation Number
\newcommand{\athecalnum}{A\thecalnum}
\newcommand{\calref}[1]{C\ref{#1}}
\newcounter{outputnum} %Output Number
\newcommand{\atheoutputnum}{A\theoutputnum}
\newcommand{\oref}[1]{O\ref{#1}}

\newcounter{goalnum} %Goal Number
\newcommand{\gthegoalnum}{GS\thegoalnum}
\newcommand{\gsref}[1]{GS\ref{#1}}
\newcounter{instnum} %Instance Number
\newcommand{\itheinstnum}{IM\theinstnum}
\newcommand{\iref}[1]{IM\ref{#1}}
\newcounter{reqnum} %Requirement Number
\newcommand{\rthereqnum}{P\thereqnum}
\newcommand{\rref}[1]{R\ref{#1}}
\newcounter{lcnum} %Likely change number
\newcommand{\lthelcnum}{LC\thelcnum}
\newcommand{\lcref}[1]{LC\ref{#1}}

\newcommand{\famname}{FSL} % PUT YOUR PROGRAM NAME HERE

\usepackage{fullpage}

\begin{document}

\title{\famname\\A Fourier Series Library} 
\author{Bo Cao}
\date{\today}

\maketitle

~\newpage

\pagenumbering{roman}

\section{Revision History}

\begin{tabularx}{\textwidth}{p{3cm}p{2cm}X}
\toprule {\bf Date} & {\bf Version} & {\bf Notes}\\
\midrule
Oct. 10, 2019 & 0.9 & Draft submitted for review by Dr. Smith\\
TBD & 1.0 & Revised in accordance with advice from Dr. Smith\\
\bottomrule
\end{tabularx}

~\newpage
	
\section{Reference Material}

This section records information for easy reference.

\subsection{Table of Units}

This section is not applicable, due to the fact that this library is a pure mathematical computation library.


\subsection{Table of Symbols}

The table that follows summarizes the symbols used in this document. 
The choice of symbols is made to be consistent with the Fourier 
series literature and with existing documentation for Fourier series 
libraries.

\renewcommand{\arraystretch}{1.2}
\noindent \begin{longtable*}{l l p{12cm}} \toprule
\textbf{symbol} & \textbf{description}\\
\midrule 
$f(t)$, $g(t)$, ... & $[-\pi/\omega, \pi/\omega]\rightarrow\mathbb{R}$ 
functions of $t$, where $\omega$ is defined as below\\
$\mathbb{R}$ & Set of real numbers\\
$\mathbb{Z}$ & Set of integers\\
$[-\pi/\omega, \pi/\omega]$ & Set of real numbers that are neither smaller 
than $-\pi/\omega$, nor greater than $\pi/\omega$ \\
$\rightarrow$ & Indicate mapping\\
$\mathbb{A}^{*}$ & Non-negative subset of set $\mathbb{A}$ (either $\mathbb{R}$ 
or $\mathbb{Z}$)\\
$\mathbb{A}^{+}$ & Positive subset of set $\mathbb{A}$ (either $\mathbb{R}$ 
or $\mathbb{Z}$)\\
$\sum_{i=m_1}^{m_2}a_i$ & Summation of $a_i, i=m_1, m_1+1, ..., m_2$\\
$n$ & Length of cut-off Fourier series\\ 
$\omega$ & Base frequency of Fourier series \\
$\mathit{CFSf}$, $\mathit{CFSg}$ & $\mathit{CFS}(f(t), n, \omega)$ and 
$\mathit{CFS}(g(t), n, \omega)$ respectively.\\
\bottomrule
\end{longtable*}

\subsection{Abbreviations and Acronyms}

\renewcommand{\arraystretch}{1.2}
\begin{tabular}{l l} 
  \toprule		
  \textbf{symbol} & \textbf{description}\\
  \midrule 
  A & Assumption\\
  API & Application Programming Interface\\
  CA & Commonality Analysis\\
  CFS & Cut-off Fourier Series\\
  DD & Data Definition\\
  FS & Fourier Series\\
  GD & General Definition\\
  GS & Goal Statement\\
  IFS & Infinite Fourier Series\\
  IM & Instance Model\\
  LC & Likely Change\\
  PS & Physical System Description\\
  R & Requirement\\
  \famname{} & Fourier Series Library.\\
  T & Theoretical Model\\
  \bottomrule
\end{tabular}\\


\newpage

\tableofcontents

~\newpage

\pagenumbering{arabic}

\section{Introduction}

\subsection{Purpose of Document}
This document introduces a library, \famname, mainly designed and implemented 
for Fourier series related calculations. This document is based on the 
template introduced in \cite{Smith2006}, \cite{SmithAndLai2005}, 
and \cite{SmithMcCutchanAndCarette2017}.

\subsection{Scope of the Family} 
The scope of the family is limited to calculations related to the Fourier 
series of mathematical functions.

\subsection{Characteristics of Intended Reader} 
The intended readers must have the following knowledge.
\begin{itemize}
	\item Deep knowledge of Fourier series (found in advanced 
	calculus and real analysis courses);
	\item Computational error analysis (found in introductory numerical 
	analysis classes); and
	\item Detailed knowledge of the programming language utilized 
	in the implementation of this library.	
\end{itemize}

\subsection{Organization of Document}
This document consists of the following sections.
\begin{itemize}
	\item \autoref{Sc:GSD} generally describes this library, including 
	what it is, who uses it, and what kinds of environment it needs.
	\item \autoref{Sc:CA} introduces the common elements of this library. 
	This section overviews its background, defines related data, 
	introduces theories needed for calculation, 
	and states the goals of this library.
	\item \autoref{Sc:Var} covers the different functions of this library, 
	and states its variability in assumptions, calculations, and outputs.
	\item \autoref{Sc:Req} informs users, developers, maintainers, 
	and reviewers of this library, about its requirements, 
	both functional ones and nonfunctional ones.
	\item \autoref{Sc:LC} states the changes of this library that are 
	likely in the future.
	\item \autoref{Sc:Trace} shows the traceability matrix of this library, 
	to inform the developers of their changes' impacts.
	\item \autoref{Appendix:Operations} gives proof of the 
	operation-related theories in \autoref{Sc:CA}.
\end{itemize}

\section{General System Description}\label{Sc:GSD}

This section identifies the interfaces between the system and its environment,
describes the potential user characteristics and lists the potential system
constraints.

\subsection{Potential System Contexts}

The users provide the data for the library to  calculate, and the library 
returns the calculation results.
\begin{itemize}
\item User Responsibilities:
\begin{itemize}
\item Provide the inputs of the functions in the library.
\item When needed, allocate memory space, and pass their location to the library, so that the library can store the output in this user-provided space.
\item For the conversion from mathematical functions to Fourier series, 
ensure that provided mathematical functions are eligible for transformation. 
\end{itemize}
\item \famname{} Responsibilities:
\begin{itemize}
\item Detect data type mismatch, such as a string of characters 
instead of a floating point number.
\item Unless otherwise mentioned in this document, detect illegality 
of input values, such as mismatched length of Fourier series 
for binary operations.
\item Unless otherwise mentioned, manage the memory space 
required by this library.
\end{itemize}
\end{itemize}

\subsection{Potential User Characteristics} \label{SecUserCharacteristics}

The end user of \famname{} should have an understanding of undergraduate 
level 2 Calculus and/or Real Analysis, undergraduate level 1 Numerical 
Analysis, and any one of the programming languages, in which this library 
provides a set of APIs.

\subsection{Potential System Constraints}
The potential systems must contain the compilation/interpretation environment 
for the programming languages in which this library is developed, as well 
as dependent libraries.

\section{Commonalities}\label{Sc:CA}

\subsection{Background Overview} \label{Sbsc:CA-Background}
Fourier series are used in lots of analysis, analysis of signals as an example 
\cite{papoulis1977signal}. The Fourier series of a function mainly uses 
a linear combination of 'sin' and 'cos' functions to represent and approximate 
this function. In real-life research, we sometimes need to perform Fourier 
series related operations. For example, when we need to analyze the total 
effects of two signals, each approximated by its Fourier series, 
we need to calculated the summation of these two Fourier series. 
Or, when we need to calculate the strength of a signal from its Fourier 
series, we need to calculate this Fourier series' amplitude.
\subsection{Terminology and Definitions}

This subsection provides a list of terms that are used in the subsequent
sections and their meaning, with the purpose of reducing ambiguity and making it
easier to correctly understand the requirements:

\begin{itemize}

\item Binary operations: an operation that involves two operands.
\item Fourier series: a series of real numbers acting as weights in the weighed sum of 'sin' and 'cos' functions, to represent/approximate a function.
\item Amplitude: a map $\mathit{CFS}\rightarrow \mathbb{R}^{*}$ to measure this CFS's strength.
\end{itemize}
\an{This section needs further expansion.}
\subsection{Data Definitions} \label{sec_datadef}

This section collects and defines all the data needed to build the instance
models.
~\newline

\noindent
\begin{minipage}{\textwidth}
\renewcommand*{\arraystretch}{1.5}
\begin{tabular}{| p{\colAwidth} | p{\colBwidth}|}
\hline
\rowcolor[gray]{0.9}
Number& DD\refstepcounter{datadefnum}\thedatadefnum \label{DD:IFS}\\
\hline
Label& \bf Infinite Fourier Series (IFS) of mathematical functions\\
\hline
Symbol &$\mathit{IFS}$\\
\hline
  Equation&$\mathit{IFS}(f(t), \omega) = \big[A_{inf, i}, i=0, 1, ...; B_{inf, i}, 
  i=1, 2, ...\big]$, satisfying $f(t)=\sum_{i=0}^{+\infty}A_{inf, i}
  \cos(i\omega t)+\sum_{i=1}^{+\infty}B_{inf, i}\sin(i\omega t)$\\
  \hline
  Description & In this equation, \begin{itemize}
  	\item $f(t)$ is an $[-\pi/\omega, \pi/\omega]]
  	\rightarrow\mathbb{R}$ function, whose IFS exists;
  	\item $\omega$ is the base frequency of this IFS, 
  	and it is designated by the user;
  	\item $A_{inf, i}\in\mathbb{R}, i=0, 1, ...$; and
  	\item $B_{inf, i}\in\mathbb{R}, i=1, 2, ...$.
  \end{itemize}  \\
  \hline
  Sources& Tolstov, G.P. and Silverman, R.A., Fourier Series, 
  Dover Books on Mathematics, 1976, Dover Publications, 
  \cite{tolstov1976fourier}\\
  \hline
  Ref.\ By & \iref{DD:CFS}, \aref{Ass:FunctionProperty}, \calref{Cal:Normal}, 
  \calref{Cal:Error}, \oref{Output:Faithful},  and \oref{Output:Error}\\
  \hline
\end{tabular}
\end{minipage}\\
~\newline

\noindent
\begin{minipage}{\textwidth}
	\renewcommand*{\arraystretch}{1.5}
	\begin{tabular}{| p{\colAwidth} | p{\colBwidth}|}
		\hline
		\rowcolor[gray]{0.9}
		Number& DD\refstepcounter{datadefnum}\thedatadefnum \label{DD:CFS}\\
		\hline
		Label& \bf Cut-off Fourier Series (CFS) of mathematical functions\\
		\hline
		Symbol &$\mathit{CFS}$\\
		\hline
		Equation&$\mathit{CFS}(f(t), n, \omega)=\big[A_i, i=0, 1, ..., n; 
		B_i, i=1, 2, ..., n\big]$, satisfying $A_i=A_{inf, i}, i=0, 1, ..., n; 
		B_i=B_{inf, i}, i=1, 2, ..., n$, where 
		$\omega$, $A_{inf, i}$'s 
		and $B_{inf, i}$'s come from $\mathit{IFS}(f(t), \omega)$ \\
		\hline
		Description & $n\in \mathbb{Z}^{*}$ is called cut-off length.\\
		\hline
		Sources& Defined by author\\
		\hline
		Ref.\ By & \tref{T:Transformation}, \tref{T:Addition}, 
		\tref{T:Subtraction}, \tref{T:Multiplication}, 
		\tref{T:Division}, \tref{T:Amplitude}, 
		\aref{Ass:FunctionProperty},  \calref{Cal:Normal}, 
		\calref{Cal:Error}, \calref{Cal:Memory}, 
		\oref{Output:Faithful}, and \oref{Output:Error}\\
		\hline
	\end{tabular}
\end{minipage}\\
~\newline

From now on, denote $A_i$'s and $B_i$'s in $\mathit{CFSf}$ 
as $A_{f, i}$'s and $B_{f, i}$'s respectively, and same for 
$A_{g, i}$'s and $B_{g, i}$'s.
~\newline

\noindent
\begin{minipage}{\textwidth}
	\renewcommand*{\arraystretch}{1.5}
	\begin{tabular}{| p{\colAwidth} | p{\colBwidth}|}
		\hline
		\rowcolor[gray]{0.9}
		Number& DD\refstepcounter{datadefnum}\thedatadefnum 
		\label{DD:Approximation}\\
		\hline
		Label& \bf Approximation of function values from CFS\\
		\hline
		Symbol &$\mathit{App}$\\
		\hline
		Equation& $\mathit{App}(\mathit{CFSf}, t_1)=\sum_{i=0}^{n}A_{f,i}\cos(i\omega t_1)+
		\sum_{i=1}^{n}B_{f,i}\sin(i\omega t_1)$ \\
		\hline
		Description & Define approximated function value of a function 
		calculated from its CFS.\\
		\hline
		Sources& Defined by author\\
		\hline
		Ref.\ By & \iref{IM:Addition}, \calref{Cal:Normal}, 
		\calref{Cal:Error}, \oref{Output:Faithful}, and \oref{Output:Error}\\
		\hline
	\end{tabular}
\end{minipage}\\
~\newline

\noindent
\begin{minipage}{\textwidth}
	\renewcommand*{\arraystretch}{1.5}
	\begin{tabular}{| p{\colAwidth} | p{\colBwidth}|}
		\hline
		\rowcolor[gray]{0.9}
		Number& DD\refstepcounter{datadefnum}\thedatadefnum \label{DD:Addition}\\
		\hline
		Label& \bf Addition of CFS's\\
		\hline
		Symbol &$+$\\
		\hline
		Equation& $\mathit{CFSf}+\mathit{CFSg}=\mathit{CFS}(f(t)+g(t), 
		n, \omega)$ \\
		\hline
		Description & Define the rule of addition of two CFS's\\
		\hline
		Sources& Defined by author\\
		\hline
		Ref.\ By & \iref{IM:Addition}, \calref{Cal:Normal}, 
		\calref{Cal:Error}, \oref{Output:Faithful}, 
		and \oref{Output:Error}\\
		\hline
	\end{tabular}
\end{minipage}\\
~\newline

\noindent
\begin{minipage}{\textwidth}
	\renewcommand*{\arraystretch}{1.5}
	\begin{tabular}{| p{\colAwidth} | p{\colBwidth}|}
		\hline
		\rowcolor[gray]{0.9}
		Number& DD\refstepcounter{datadefnum}\thedatadefnum 
		\label{DD:Subtraction}\\
		\hline
		Label& \bf Subtraction of CFS's\\
		\hline
		Symbol &$-$\\
		\hline
		Equation& $\mathit{CFSf}-\mathit{CFSg}=
		\mathit{CFS}(f(t)-g(t), n, \omega)$ \\
		\hline
		Description & Define the rule of subtraction of two CFS's\\
		\hline
		Sources& Defined by author\\
		\hline
		Ref.\ By & \iref{IM:Subtraction}, \calref{Cal:Normal}, 
		\calref{Cal:Error}, \oref{Output:Faithful}, and \oref{Output:Error}\\
		\hline
	\end{tabular}
\end{minipage}\\
~\newline

\noindent
\begin{minipage}{\textwidth}
	\renewcommand*{\arraystretch}{1.5}
	\begin{tabular}{| p{\colAwidth} | p{\colBwidth}|}
		\hline
		\rowcolor[gray]{0.9}
		Number& DD\refstepcounter{datadefnum}\thedatadefnum 
		\label{DD:Multiplication}\\
		\hline
		Label& \bf Multiplication of CFS's\\
		\hline
		Symbol &$*$\\
		\hline
		Equation& $\mathit{CFSf}*\mathit{CFSg}=
		\mathit{CFS}(f(t)*g(t), n, \omega)$ \\
		\hline
		Description & Define the rule of multiplication of two CFS's\\
		\hline
		Sources& Defined by author\\
		\hline
		Ref.\ By & \iref{IM:Multiplication}, \calref{Cal:Normal}, 
		\calref{Cal:Error}, \oref{Output:Faithful}, 
		and \oref{Output:Error}\\
		\hline
	\end{tabular}
\end{minipage}\\
~\newline

\noindent
\begin{minipage}{\textwidth}
	\renewcommand*{\arraystretch}{1.5}
	\begin{tabular}{| p{\colAwidth} | p{\colBwidth}|}
		\hline
		\rowcolor[gray]{0.9}
		Number& DD\refstepcounter{datadefnum}\thedatadefnum 
		\label{DD:Division}\\
		\hline
		Label& \bf Division of CFS's\\
		\hline
		Symbol &$/$\\
		\hline
		Equation& $\mathit{CFSf}/\mathit{CFSg}=
		\mathit{CFS}(f(t)/g(t), n, \omega)$ \\
		\hline
		Description & Define the rule of division of two CFS's\\
		\hline
		Sources& Defined by author\\
		\hline
		Ref.\ By & \iref{IM:Division}, \calref{Cal:Normal}, 
		\calref{Cal:Error},  \oref{Output:Faithful}, 
		and \oref{Output:Error}\\
		\hline
	\end{tabular}
\end{minipage}\\
~\newline

\noindent
\begin{minipage}{\textwidth}
	\renewcommand*{\arraystretch}{1.5}
	\begin{tabular}{| p{\colAwidth} | p{\colBwidth}|}
		\hline
		\rowcolor[gray]{0.9}
		Number& DD\refstepcounter{datadefnum}\thedatadefnum 
		\label{DD:Function}\\
		\hline
		Label& \bf Function of CFS's\\
		\hline
		Symbol &$g(\mathit{CFS}(f(t), n, \omega))$\\
		\hline
		Equation& $g(\mathit{CFS}(f(t), n, \omega))=\mathit{CFS}(g(f(t)), n, \omega)$.\\
		\hline
		Description & Define the rule of the function of a CFS\\
		\hline
		Sources& Defined by author\\
		\hline
		Ref.\ By & \iref{IM:Function}, \aref{Ass:BasicFunction}, 
		\calref{Cal:Normal}, and \oref{Output:Faithful}\\
		\hline
	\end{tabular}
\end{minipage}\\
~\newline

\noindent
\begin{minipage}{\textwidth}
	\renewcommand*{\arraystretch}{1.5}
	\begin{tabular}{| p{\colAwidth} | p{\colBwidth}|}
		\hline
		\rowcolor[gray]{0.9}
		Number& DD\refstepcounter{datadefnum}\thedatadefnum 
		\label{DD:Amplitude}\\
		\hline
		Label& \bf Amplitude of a CFS\\
		\hline
		Symbol &$\mathit{Amp}(\mathit{CFSf})$\\
		\hline
		Equation& \begin{equation}
			\mathit{Amp}(\mathit{CFSf})=\sqrt{\frac{\omega}
			{2\pi}\int_{-\pi/\omega}^{\pi/\omega}
			\mathit{App}^2(\mathit{CSFf}, t_1)\textit{d}t_1}
		\end{equation}.\\
		\hline
		Description & Define the amplitude/size of a CFS\\
		\hline
		Sources& Defined by author\\
		\hline
		Ref.\ By & \iref{IM:Amplitude}, \aref{Ass:FunctionProperty}, 
		\calref{Cal:Normal}, and \oref{Output:Faithful}\\
		\hline
	\end{tabular}
\end{minipage}\\
~\newline

\noindent
\begin{minipage}{\textwidth}
	\renewcommand*{\arraystretch}{1.5}
	\begin{tabular}{| p{\colAwidth} | p{\colBwidth}|}
		\hline
		\rowcolor[gray]{0.9}
		Number& DD\refstepcounter{datadefnum}\thedatadefnum 
		\label{DD:Equality}\\
		\hline
		Label& \bf Tolerated equality of two CFS's\\
		\hline
		Symbol &$=$\\
		\hline
		Equation& $\mathit{CFSf}=\mathit{CFSg}$ is the same as $\mathit{Amp}
		(\mathit{CFSf}-\mathit{CFSg})\leq \mathit{err}$, 
		where $\mathit{err}$ is an user-given tolerance.\\
		\hline
		Description & Define whether two CFS's are equal within a given 
		tolerance\\
		\hline
		Sources& Defined by author\\
		\hline
		Ref.\ By & \iref{IM:ToleratedEquality}, \calref{Cal:Normal}, 
		and \oref{Output:Faithful}\\
		\hline
	\end{tabular}
\end{minipage}\\
~\newline
\subsection{Goal Statements}

\noindent Given the corresponding inputs, the goal statements are:

\begin{itemize}

\item[GS\refstepcounter{goalnum}\thegoalnum \label{GS:ConvertFromFunc}:] 
When given a function $f(t)$, the cut-off length $n$, and a base frequency 
$\omega$, return the function's CFS $\mathit{CFS}(f(t), n, \omega)$.
\item[GS\refstepcounter{goalnum}\thegoalnum \label{GS:FuncValue}:] 
When given a CFS $\mathit{CFS}(f(t), n, \omega)$ and a value of $t$ as $t_1$, 
return the approximated value of $f(t_1)$ computed from the given values.
\item[GS\refstepcounter{goalnum}\thegoalnum \label{GS:Operation}:] 
When given two CFS's $\mathit{CFSf}$, $\mathit{CFSg}$, and one of the 
operations addition, subtraction, multiplication, and division, return 
the result of this operation on these two CFS's.
\item[GS\refstepcounter{goalnum}\thegoalnum \label{GS:Function}:] 
When given a CFS $\mathit{CFSf}$, and a function $g(t)$ from the basic 
function sets defined by this library, return the CFS $g(CFSf)$. 
\an{I don't know whether to define the basic function set here or not. 
I can, and I don't see any possible changes}
\item[GS\refstepcounter{goalnum}\thegoalnum \label{GS:ConvertFromOther}:] 
When given a set of values including $n$, $\omega$, $A_i$'s, and $B_i$'s, 
return the CFS built on these values.
\item[GS\refstepcounter{goalnum}\thegoalnum \label{GS:ConvertToOther}:] 
When given a CFS, store its values of $n$, $\omega$, $A_i$'s, and $B_i$'s 
in the user-designated spaces.
\item[GS\refstepcounter{goalnum}\thegoalnum \label{GS:Amp}:] When given 
a CFS, return its amplitude.
\item[GS\refstepcounter{goalnum}\thegoalnum \label{GS:ToleratedEquality}:] 
When given two CFS's and a tolerance, return whether they are equal 
within the tolerance.
\end{itemize}

\subsection{Theoretical Models} \label{sec_theoretical}

This section focuses on the general equations and laws that \famname{} is based
on.
~\newline

\noindent
\begin{minipage}{\textwidth}
\renewcommand*{\arraystretch}{1.5}
\begin{tabular}{| p{\colAwidth} | p{\colBwidth}|}
  \hline
  \rowcolor[gray]{0.9}
  Number& T\refstepcounter{theorynum}\thetheorynum 
  \label{T:Transformation}\\
  \hline
  Label&\bf Fourier Transformation\\
  \hline
  Equation&  
  \begin{equation}
  \label{Eq:DFT}
  	\begin{aligned}
  	A_0 &=\frac{\omega}{2\pi}\int_{-\pi/\omega}^{\pi/\omega}f(t); \\
  	A_i &=\frac{\omega}{\pi}\int_{-\pi/\omega}^{\pi/\omega}f(t)\cos(i\omega t),
  	~i=1:n; \\
  	B_i &=\frac{\omega}{\pi}\int_{-\pi/\omega}^{\pi/\omega}f(t)\sin(i\omega t),
  	~i=1:n. \\
  	\end{aligned}
  \end{equation}\\
  \hline
  Description & The above equation calculates $A_{inf,i}$'s and $B_{inf,i}$'s, 
  thus $A_i (i=0:n)$, $B_i(i=1:n)$ in $\mathit{CFSf}$ from given $f(t)$, $n$ 
  and $\omega$.\\
                
  \hline
  Source & Tolstov, G.P. and Silverman, R.A., Fourier Series, Dover 
  Books on Mathematics, 1976, Dover Publications, 
  \cite{tolstov1976fourier}\\
  \hline
  Ref.\ By & \aref{Ass:FunctionProperty}, \calref{Cal:Normal}, 
  \calref{Cal:Error}, \oref{Output:Faithful}, and \oref{Output:Error}\\
  \hline
\end{tabular}
\end{minipage}\\

~\newline
\noindent
\begin{minipage}{\textwidth}
	\renewcommand*{\arraystretch}{1.5}
	\begin{tabular}{| p{\colAwidth} | p{\colBwidth}|}
		\hline
		\rowcolor[gray]{0.9}
		Number& T\refstepcounter{theorynum}\thetheorynum 
		\label{T:Addition}\\
		\hline
		Label&\bf Addition of two CFS's\\
		\hline
		Equation&  
		\begin{equation}
		\begin{aligned}
		A_{f+g, i}&=A_{f, i} + A_{g, i},~i=0:n\\
		B_{f+g, i}&=B_{f, i} + B_{g, i},~i=1:n\\
		\end{aligned}
		\end{equation}\\
		\hline
		Description & The above equation calculates $A_i (i=0:n)$, $B_i(i=1:n)$ in $\mathit{CFS}(f(t)+g(t), n, \omega)$ (represented by $A_{f+g, i}$ and $B_{f+g, i}$ respectively) from $\mathit{CFSf}$ and $\mathit{CFSg}$.\\
		
		\hline
		Source & Developed by author in \autoref{Appendix:Operations}\\
		\hline
		Ref.\ By & \aref{Ass:CFSPropertyMatch}, \calref{Cal:Normal}, 
		and \oref{Output:Faithful}\\
		\hline
	\end{tabular}
\end{minipage}\\
~\newline
\noindent
\begin{minipage}{\textwidth}
	\renewcommand*{\arraystretch}{1.5}
	\begin{tabular}{| p{\colAwidth} | p{\colBwidth}|}
		\hline
		\rowcolor[gray]{0.9}
		Number& T\refstepcounter{theorynum}\thetheorynum 
		\label{T:Subtraction}\\
		\hline
		Label&\bf Subtraction of two CFS's\\
		\hline
		Equation&  
		\begin{equation}
		\begin{aligned}
		A_{f-g, i}
		&=A_{f, i} - A_{g, i},~i=0:n\\
		B_{f-g, i}
		&=B_{f, i} - B_{g, i},~i=1:n\\
		\end{aligned}
		\end{equation}\\
		\hline
		Description & The above equation calculates 
		$A_i(i=0:n)$, $B_i(i=1:n)$ in 
		$\mathit{CFS}(f(t)-g(t), n, \omega)$ 
		(represented by $A_{f-g, i}$ and $B_{f-g, i}$ respectively) 
		from $\mathit{CFSf}$ and $\mathit{CFSg}$.\\
		\hline
		Source & Developed by author in \autoref{Appendix:Operations}\\
		\hline
		Ref.\ By & \aref{Ass:CFSPropertyMatch}, \calref{Cal:Normal}, 
		and \oref{Output:Faithful}\\
		\hline
	\end{tabular}
\end{minipage}\\

~\newline

\noindent
\begin{minipage}{\textwidth}
	\renewcommand*{\arraystretch}{1.5}
	\begin{tabular}{| p{\colAwidth} | p{\colBwidth}|}
		\hline
		\rowcolor[gray]{0.9}
		Number& T\refstepcounter{theorynum}\thetheorynum \label{T:Multiplication}\\
		\hline
		Label&\bf Multiplication of two CFS's\\
		\hline
		Equation&  
		\begin{equation}
		\begin{aligned}
		A_{f*g, i}&=\frac{1}{2}\sum_{j=0}^{n-i}(A_{f,i}A_{g,i+j}
		+A_{f, i+j}A_{g, i}+B_{f,i}B_{g,i+j}+B_{f,i+j}B_{g,i})\\
		&+\frac{1}{2}\sum_{j=0}^{i}(A_{f,j}A_{g,i-j}-B_{f,j}B_{g,i-j}),
		~i=0:n\\
		B_{f*g, i}&=\frac{1}{2}\sum_{j=0}^{n-i}(A_{f,i}B_{g,i+j}
		+A_{f, i+j}B_{g, i}+B_{f,i}A_{g,i+j}+B_{f,i+j}A_{g,i})\\
		&+\frac{1}{2}\sum_{j=0}^{i}(A_{f,j}B_{g,i-j}+B_{f,j}A_{g,i-j}),
		~i=1:n\\
		\end{aligned}
		\end{equation}\\
		\hline
		Description & The above equation calculates $A_i(i=0:n)$, 
		$B_i(i=1:n)$ in $\mathit{CFS}(f(t)*g(t), n, \omega)$ 
		(represented by $A_{f*g, i}$ and $B_{f*g, i}$ respectively) 
		from $\mathit{CFSf}$ and $\mathit{CFSg}$.\\
		\hline
		Source & Developed by author in \autoref{Appendix:Operations}\\
		\hline
		Ref.\ By & \aref{Ass:CFSPropertyMatch}, \calref{Cal:Normal}, 
		and \oref{Output:Faithful}\\
		\hline
	\end{tabular}
\end{minipage}\\

~\newline
\noindent
\begin{minipage}{\textwidth}
	\renewcommand*{\arraystretch}{1.5}
	\begin{tabular}{| p{\colAwidth} | p{\colBwidth}|}
		\hline
		\rowcolor[gray]{0.9}
		Number& T\refstepcounter{theorynum}\thetheorynum 
		\label{T:Division}\\
		\hline
		Label&\bf Division of two CFS's\\
		\hline
		Equation& Solve the following equations for $A_{f/g, i}$'s 
		and $B_{f/g, i}$'s.   
		\begin{equation}
		\begin{aligned}
		A_{f, i}&=\frac{1}{2}\sum_{j=0}^{n-i}(A_{f/g,i}A_{g,i+j}
		+A_{f/g, i+j}A_{g, i}+B_{f/g,i}B_{g,i+j}+B_{f/g,i+j}B_{g,i})\\
		&+\frac{1}{2}\sum_{j=0}^{i}(A_{f/g,j}A_{g,i-j}-B_{f/g,j}B_{g,i-j}),
		~i=0:n\\
		B_{f, i}&=\frac{1}{2}\sum_{j=0}^{n-i}(A_{f/g,i}B_{g,i+j}
		+A{f/g, i+j}B_{g, i}+B_{f/g,i}A_{g,i+j}+B_{f/g,i+j}A_{g,i})\\
		&+\frac{1}{2}\sum_{j=0}^{i}(A_{f/g,j}B_{g,i-j}+B_{f/g,j}A_{g,i-j}),
		~i=1:n\\
		\end{aligned}
		\end{equation}\\
		\hline
		Description & We solve the above equations for 
		$A_i(i=0:n)$, $B_i(i=1:n)$ in $\mathit{CFS}(f(t)/g(t), n, \omega)$ 
		(represented by $A_{f/g, i}$'s and $B_{f/g, i}$'s respectively) 
		from given $\mathit{CFSf}$ and $\mathit{CFSg}$.\\
		\hline
		Source & Developed by author in \autoref{Appendix:Operations}\\
		\hline
		Ref.\ By & \aref{Ass:CFSPropertyMatch}, \calref{Cal:Normal}, 
		\calref{Cal:Error}, \oref{Output:Faithful}, 
		and \oref{Output:Error}\\
		\hline
	\end{tabular}
\end{minipage}\\

~\newline
\noindent
\begin{minipage}{\textwidth}
	\renewcommand*{\arraystretch}{1.5}
	\begin{tabular}{| p{\colAwidth} | p{\colBwidth}|}
		\hline
		\rowcolor[gray]{0.9}
		Number& T\refstepcounter{theorynum}\thetheorynum 
		\label{T:Function}\\
		\hline
		Label&\bf Function of a CFS\\
		\hline
		Equation&$g(\mathit{CFS}(f(t), n, \omega))=\sum_{i=0}^{n}a_i 
		\mathit{CFS}^i(f(t), n, \omega)$\\
		\hline
		Description & The above equation calculates the result of a function being applied to a CFS. $a_i, i=0:n$ is 
		the first $(n+1)$ coefficients of the Taylor series of $g(t)$, and 
		the $n$-th ($n\in\mathbb{Z}^{+}$) power of CFS is defined as $n$ 
		copies of this CFS multiplied together. ($0$-th power is defined 
		as the CFS of $f(t)=1$.)\\
		\hline
		Source & Developed by author in \autoref{Appendix:Operations}\\
		\hline
		Ref.\ By & \calref{Cal:Normal} and \oref{Output:Faithful}\\
		\hline
	\end{tabular}
\end{minipage}\\
~\newline

\noindent
\begin{minipage}{\textwidth}
	\renewcommand*{\arraystretch}{1.5}
	\begin{tabular}{| p{\colAwidth} | p{\colBwidth}|}
		\hline
		\rowcolor[gray]{0.9}
		Number& T\refstepcounter{theorynum}\thetheorynum 
		\label{T:Amplitude}\\
		\hline
		Label&\bf Amplitude of a CFS\\
		\hline
		Equation&  
		\begin{equation}
		\mathit{Amp}(CSFf) = \sqrt{A_0^2+\frac{1}{2}
		\sum_{i=1}^{n}(A_i^2+B_i^2)}
		\end{equation}\\
		\hline
		Description & The above equation calculates $\mathit{Amp}
		(\mathit{CFSf})$  from $\mathit{CFSf}$, especially $A_i$'s 
		and $B_i$'s in it.\\
		\hline
		Source & Developed by author in \autoref{Appendix:Operations}\\
		\hline
		Ref.\ By & \calref{Cal:Normal} and \oref{Output:Faithful}\\
		\hline
	\end{tabular}
\end{minipage}\\
~\newline

\section{Variabilities}\label{Sc:Var}
\subsection{Instance Models}
\noindent
\begin{minipage}{\textwidth}
	\renewcommand*{\arraystretch}{1.5}
	\begin{tabular}{| p{\colAwidth} | p{\colBwidth}|}
		\hline
		\rowcolor[gray]{0.9}
		Number& IM\refstepcounter{instnum}\theinstnum 
		\label{IM:CFScoeff}\\
		\hline
		Label& \bf Calculate coefficients of CFS's\\
		\hline
		Input& $f(t)$, a $[-\pi/\omega, \pi/\omega]
		\rightarrow\mathbb{R}$ function\\
			& $n\in\mathbb{Z}^{*}$\\
			& $\omega\in\mathbb{R}^{+}$\\
		\hline
		Output& An object $\mathit{CFSf}$, containing the 
		following data:\\
		&$n$, $\omega$: same as the input\\
		&$A_i (i=0:n)$, $B_i (i=1:n)$: using the theory 
		\tref{T:Transformation}\\
		\hline
		Description&Input:\\
		&$f(t)$: function to be transformed into 
		Fourier series\\
		&Input \& Output:\\
		& $n$: cut-off length\\
		& $\omega$: base frequency\\
		&Output:\\
		&$A_i$'s and $B_i$'s: coefficients in $\mathit{CFSf}$\\
		\hline
		Sources&Same as \tref{T:Transformation}		\\
		\hline
		Ref.\ By & \aref{Ass:FunctionProperty}, 
		\calref{Cal:Normal}, \calref{Cal:Error}, 
		\oref{Output:Faithful}, and \oref{Output:Error}\\
		\hline
	\end{tabular}
\end{minipage}\\
~\newline
\noindent
\begin{minipage}{\textwidth}
	\renewcommand*{\arraystretch}{1.5}
	\begin{tabular}{| p{\colAwidth} | p{\colBwidth}|}
		\hline
		\rowcolor[gray]{0.9}
		Number& IM\refstepcounter{instnum}\theinstnum 
		\label{IM:Addition}\\
		\hline
		Label& \bf Addition of two CFS's \\
		\hline
		Input& $\mathit{CFSf}$, $\mathit{CFSg}$\\
		\hline
		Output& $\mathit{CFS}(f(t)+g(t), n, \omega)$, whose $A_i$'s 
		and $B_i$'s are computed using the theory \tref{T:Addition}\\
		\hline
		Description&Input:\\
		&$\mathit{CFSf}$, $\mathit{CFSg}$: operands of the addition\\
		&Output:\\
		& $\mathit{CFS}(f(t)+g(t), n, \omega)$: result of the addition\\
		\hline
		Sources&Same as \tref{T:Addition}		\\
		\hline
		Ref.\ By & \aref{Ass:CFSPropertyMatch}, \calref{Cal:Normal}, 
		and \oref{Output:Faithful}\\
		\hline
	\end{tabular}
\end{minipage}\\
~\newline

\noindent
\begin{minipage}{\textwidth}
	\renewcommand*{\arraystretch}{1.5}
	\begin{tabular}{| p{\colAwidth} | p{\colBwidth}|}
		\hline
		\rowcolor[gray]{0.9}
		Number& IM\refstepcounter{instnum}\theinstnum 
		\label{IM:Subtraction}\\
		\hline
		Label& \bf Subtraction of two CFS's \\
		\hline
		Input& $\mathit{CFSf}$, $\mathit{CFSg}$\\
		\hline
		Output& $\mathit{CFS}(f(t)-g(t), n, \omega)$, whose $A_i$'s 
		and $B_i$'s are computed using the theory \tref{T:Subtraction}\\
		\hline
		Description&Input:\\
		&$\mathit{CFSf}$, $\mathit{CFSg}$: operands of the subtraction\\
		&Output:\\
		& $\mathit{CFS}(f(t)-g(t), n, \omega)$: result of the subtraction\\
		\hline
		Sources&Same as \tref{T:Subtraction}		\\
		\hline
		Ref.\ By & \aref{Ass:CFSPropertyMatch}, \calref{Cal:Normal}, 
		and \oref{Output:Faithful}\\
		\hline
	\end{tabular}
\end{minipage}\\
~\newline

\noindent
\begin{minipage}{\textwidth}
	\renewcommand*{\arraystretch}{1.5}
	\begin{tabular}{| p{\colAwidth} | p{\colBwidth}|}
		\hline
		\rowcolor[gray]{0.9}
		Number& IM\refstepcounter{instnum}\theinstnum 
		\label{IM:Multiplication}\\
		\hline
		Label& \bf Multiplication of two CFS's \\
		\hline
		Input& $\mathit{CFSf}$, $\mathit{CFSg}$\\
		\hline
		Output& $\mathit{CFS}(f(t)*g(t), n, \omega)$, whose $A_i$'s 
		and $B_i$'s are computed using the theory 
		\tref{T:Multiplication}\\
		\hline
		Description&Input:\\
		&$\mathit{CFSf}$, $\mathit{CFSg}$: operands of the 
		multiplication\\
		&Output:\\
		& $\mathit{CFS}(f(t)*g(t), n, \omega)$: result 
		of the multiplication\\
		\hline
		Sources&Same as \tref{T:Multiplication}\\
		\hline
		Ref.\ By & \aref{Ass:CFSPropertyMatch}, \calref{Cal:Normal}, 
		and \oref{Output:Faithful}\\
		\hline
	\end{tabular}
\end{minipage}\\
~\newline

\noindent
\begin{minipage}{\textwidth}
	\renewcommand*{\arraystretch}{1.5}
	\begin{tabular}{| p{\colAwidth} | p{\colBwidth}|}
		\hline
		\rowcolor[gray]{0.9}
		Number& IM\refstepcounter{instnum}\theinstnum 
		\label{IM:Division}\\
		\hline
		Label& \bf Division of two CFS's \\
		\hline
		Input& $\mathit{CFSf}$, $\mathit{CFSg}$\\
		\hline
		Output& $\mathit{CFS}(f(t)/g(t), n, \omega)$, whose $A_i$'s 
		and $B_i$'s are computed using the theory 
		\tref{T:Division}\\
		\hline
		Description&Input:\\
		&$\mathit{CFSf}$, $\mathit{CFSg}$: operands of 
		the division\\
		&Output:\\
		& $\mathit{CFS}(f(t)+g(t), n, \omega)$: result of 
		the division\\
		\hline
		Sources&Same as \tref{T:Division}		\\
		\hline
		Ref.\ By & \aref{Ass:CFSPropertyMatch}, 
		\calref{Cal:Normal}, \calref{Cal:Error}, 
		\oref{Output:Faithful}, and \oref{Output:Error}\\
		\hline
	\end{tabular}
\end{minipage}\\
~\newline

\noindent
\begin{minipage}{\textwidth}
	\renewcommand*{\arraystretch}{1.5}
	\begin{tabular}{| p{\colAwidth} | p{\colBwidth}|}
		\hline
		\rowcolor[gray]{0.9}
		Number& IM\refstepcounter{instnum}\theinstnum 
		\label{IM:Function}\\
		\hline
		Label& \bf Function of a CFS \\
		\hline
		Input& $\mathit{CFSf}$, $g(t)$\\
		\hline
		Output& A CFS $g(CFSf)$, whose $A_i$'s and $B_i$'s 
		are computed using the theory \tref{T:Multiplication} 
		and \tref{T:Addition}\\
		\hline
		Description&Input:\\
		& $g(t)$, a basic function chosen by user from a 
		basic function set.\\
		&$\mathit{CFSf}$: dependent variable of the function $g(t)$\\
		&Output:\\
		& $g(CFSf)$: A CFS being the computed result\\
		\hline
		Sources&Easily derived from \tref{T:Multiplication} 
		and \tref{T:Addition}\\
		\hline
		Ref.\ By & \aref{Ass:BasicFunction}, \calref{Cal:Normal}, 
		and \oref{Output:Faithful}\\
		\hline
	\end{tabular}
\end{minipage}\\
~\newline

\noindent
\begin{minipage}{\textwidth}
	\renewcommand*{\arraystretch}{1.5}
	\begin{tabular}{| p{\colAwidth} | p{\colBwidth}|}
		\hline
		\rowcolor[gray]{0.9}
		Number& IM\refstepcounter{instnum}\theinstnum 
		\label{IM:Amplitude}\\
		\hline
		Label& \bf Amplitude of a CFS \\
		\hline
		Input& $\mathit{CFSf}$\\
		\hline
		Output& $\mathit{Amp}(\mathit{CFSf})$, computed using the theory 
		\tref{T:Amplitude}\\
		\hline
		Description&Input:\\
		&$\mathit{CFSf}$: variable of the amplitude function\\
		&Output:\\
		& $\mathit{Amp}(\mathit{CFSf})$, the amplitude of $\mathit{CFSf}$\\
		\hline
		Sources&Same as \tref{T:Amplitude}		\\
		\hline
		Ref.\ By & \calref{Cal:Normal}, and  \oref{Output:Faithful}\\
		\hline
	\end{tabular}
\end{minipage}\\
~\newline

\noindent
\begin{minipage}{\textwidth}
	\renewcommand*{\arraystretch}{1.5}
	\begin{tabular}{| p{\colAwidth} | p{\colBwidth}|}
		\hline
		\rowcolor[gray]{0.9}
		Number& IM\refstepcounter{instnum}\theinstnum 
		\label{IM:ToleratedEquality}\\
		\hline
		Label& \bf Tolerated Equality Comparison of two CFS's \\
		\hline
		Input& $\mathit{CFSf}$, $\mathit{CFSg}$, $\mathit{tol}$\\
		\hline
		Output& boolean value \texttt{True} if 
		$\mathit{Amp}(\mathit{CFSf}-\mathit{CFSg})\leq \mathit{tol}$\\
		&boolean value \texttt{False} otherwise\\
		\hline
		Description&Input:\\
		&$\mathit{CFSf}$, $\mathit{CFSg}$: operands of 
		the tolerated equality comparison\\
		&$\mathit{tol}\in\mathbb{R}^{*}$\\
		&Output:\\
		& A boolean value: Whether the two operands are equal 
		within the given error tolerance\\
		&Note: The calculation of difference's amplitude 
		relies on \iref{IM:Subtraction} and \iref{IM:Amplitude}.\\
		\hline
		Sources&Same as \tref{T:Subtraction} and \tref{T:Amplitude}\\
		\hline
		Ref.\ By & \aref{Ass:CFSPropertyMatch}, \calref{Cal:Normal}, 
		and \oref{Output:Faithful}\\
		\hline
	\end{tabular}
\end{minipage}\\
~\newline

\noindent
\begin{minipage}{\textwidth}
	\renewcommand*{\arraystretch}{1.5}
	\begin{tabular}{| p{\colAwidth} | p{\colBwidth}|}
		\hline
		\rowcolor[gray]{0.9}
		Number& IM\refstepcounter{instnum}\theinstnum 
		\label{IM:ConvertTo}\\
		\hline
		Label& \bf Convert data of other structures 
		(input sources included) to a CFS \\
		\hline
		Input& $n$, $\omega$, $n+1$ real numbers $\mathit{Ain}_{i} 
		(i=0:n)$, and $n$ real numbers $\mathit{Bin}_{i} (i=1:n)$\\
		\hline
		Output& A CFS object $\mathit{CFSf}$ constructed from 
		the input data\\
		\hline
		Description&Input: Data needed for construction\\
		&Output: Constructed CSF object containing the input data\\
		\hline
		Sources&None		\\
		\hline
		Ref.\ By &  \aref{Ass:Memory}, \calref{Cal:Normal}, 
		\calref{Cal:Memory}, and \oref{Output:Faithful}\\
		\hline
	\end{tabular}
\end{minipage}\\
~\newline

\noindent
\begin{minipage}{\textwidth}
	\renewcommand*{\arraystretch}{1.5}
	\begin{tabular}{| p{\colAwidth} | p{\colBwidth}|}
		\hline
		\rowcolor[gray]{0.9}
		Number& IM\refstepcounter{instnum}\theinstnum 
		\label{IM:ConvertFrom}\\
		\hline
		Label& \bf Convert CFS to data of structures 
		(output destination included) \\
		\hline
		Input& $\mathit{CFSf}$\\
		\hline
		Output& $n$, $\omega$, $\mathit{Aout}_{i}=A_i (i=0:n)$, 
		and $\mathit{Bout}_{i}=B_i (i=1:n)$\\
		\hline
		Description&Input: The CFS object to be converted to\\
		&Output: The data extracted from the input CFS object.\\
		\hline
		Sources&None\\
		\hline
		Ref.\ By &   \aref{Ass:Memory}, \calref{Cal:Normal}, 
		\calref{Cal:Memory}, and \oref{Output:Faithful}\\
		\hline
	\end{tabular}
\end{minipage}\\
~\newline

\subsection{Assumptions}
\newcommand{\aitem}[1]{\item[A\refstepcounter{assumpnum}\theassumpnum \label{Ass:#1}:] }
\begin{itemize}

\aitem{FunctionProperty}
The functions $f(t)$ and $g(t)$ mentioned above must have definitions on 
$[-\pi/\omega, \pi/\omega]$, in which $\omega$ is given by the user, 
and must have corresponding Fourier series.
\item[A\refstepcounter{assumpnum}\theassumpnum \label{Ass:BasicFunction}:] 
Only the basic functions of CFS's are to be calculated. The aforementioned 
basic functions are those in a library-defined set.
\item[A\refstepcounter{assumpnum}\theassumpnum \label{Ass:CFSPropertyMatch}:] 
For any two-operand operations, the data $n$ and $\omega$ of these operands 
must be the same. 
\item[A\refstepcounter{assumpnum}\theassumpnum \label{Ass:Memory}:] 
User shall allocate memory space for any variable other than 
CFS's in \iref{IM:ConvertTo} and \iref{IM:ConvertFrom}.
\end{itemize}

\subsection{Calculation} \label{sec_Calculation}
\begin{itemize}
\item[C\refstepcounter{calnum}\thecalnum \label{Cal:Normal}:] 
Calculate the result based on the called function and input variables.

\end{itemize}
\subsection{Output} \label{sec_Output}    
\begin{itemize}
\item[O\refstepcounter{outputnum}\theoutputnum \label{Output:Faithful}:] 
Return the results faithfully.
\item[O\refstepcounter{outputnum}\theoutputnum \label{Output:Error}:] 
Report any detected errors to the user of the library.
\end{itemize}
\section{Requirements}\label{Sc:Req}

This section provides the functional requirements, the business tasks that the
software is expected to complete, and the nonfunctional requirements, the
qualities that the software is expected to exhibit.

\subsection{Functional Requirements}

\noindent \begin{itemize}
\newcommand{\ritem}[1]{\item[R\refstepcounter{reqnum}\thereqnum \label{R:#1}:]}
\ritem{InputDataType} 
When applicable, functions of this library accept inputs with the same 
data type for $f(t)$, $A_i$'s and/or $B_i$'s, and $\omega$. 
\ritem{OutputError} 
The error of the output must be decreasing as the $n$ in CFS's is decreasing.
\ritem{OutputDataType} 
Output the result with the same floating point data type as that of the input.
\ritem{ErrorMessage}
If the called function detects that the input variables do not 
meet the requirements, generate an error message describing 
the detected error.
\ritem{Memory} 
Manage the memory spaces required by this library 
(mainly for CFS's), and destroy them the moment they are not needed.

\end{itemize}

\subsection{Nonfunctional Requirements}

\begin{itemize}
	\item All time complexities shall be unrelated to $\omega$.
	\item The time complexity of \iref{IM:CFScoeff} shall be $O(n^2)$ 
	when the input function $f(t)$ is not complex.
	\item The time complexity of \iref{IM:Addition} and \iref{IM:Subtraction} 
	shall be $O(n)$.
	\item The time complexity of \iref{IM:Multiplication} shall be $O(n^2)$.
	\item The time complexity of \iref{IM:Division} shall be the same 
	as the best linear equation solver applicable.
	\item The time complexity of \iref{IM:Amplitude} 
	and \iref{IM:ToleratedEquality} shall be $O(n)$.
\end{itemize}

\section{Likely Changes}\label{Sc:LC}    
\noindent \begin{itemize}
\item[LC\refstepcounter{lcnum}\thelcnum\label{LC:NewOperation}:] 
We might add new operations in the future.
\item[LC\refstepcounter{lcnum}\thelcnum\label{LC:BasicFunction}:] 
The set of basic functions in \iref{IM:Function} will be 
decided in the future.
\end{itemize}

\section{Traceability Matrices and Graphs}\label{Sc:Trace}
The following matrices and graphs demonstrates the traceability 
of this project. The purpose is to provide easy references 
to the impacts on other components if a certain component is changed. 
That is, if one component has been changed, other components that share 
an 'X' with it may need change. Table \ref{Table:Traceability} 
shows the the dependencies of goals, theoretical models, 
data definitions, and instances models with the assumptions, 
calculations, and outputs.
\begin{table}
	\centering
	\begin{tabular}{|c|c|c|c|c|c|c|c|c|c|}
		\hline
		&\aref{Ass:FunctionProperty}&\aref{Ass:BasicFunction}
		&\aref{Ass:CFSPropertyMatch}&\aref{Ass:Memory}
		&\calref{Cal:Normal}&\calref{Cal:Error}
		&\calref{Cal:Memory}&\oref{Output:Faithful}
		&\oref{Output:Error}\\
		\hline
		\gsref{GS:ConvertFromFunc}
		&X& & & &X&X&  &X&X\\\hline
		\gsref{GS:FuncValue}
		& &X& & &X& & &X& \\\hline
		\gsref{GS:Operation}
		& & &X& &X&X& &X&X \\\hline
		\gsref{GS:Function}
		&X& & & &X&X& &X&X\\\hline
		\gsref{GS:ConvertFromOther}
		& & & &X&X& &X&X& \\\hline
		\gsref{GS:ConvertToOther}
		& & & &X&X& &X&X& \\\hline
		\gsref{GS:Amp}
		& & & &X& & & &X& \\\hline
		\gsref{GS:ToleratedEquality}
		& & &X& &X&X& &X&X\\\hline
		\ddref{DD:IFS}
		&X& & & &X&X& &X&X\\\hline
		\ddref{DD:CFS}
		&X& & & &X&X&X&X&X\\\hline
		\ddref{DD:Approximation}
		& & & & &X&X& &X&X\\\hline
		\ddref{DD:Addition}
		& & &X& &X&X& &X&X\\\hline
		\ddref{DD:Subtraction}
		& & &X& &X&X& &X&X\\\hline
		\ddref{DD:Multiplication}
		& & &X& &X&X& &X&X\\\hline
		\ddref{DD:Division}
		& & &X& &X&X& &X&X\\\hline
		\ddref{DD:Function}
		& &X& & &X& & &X& \\\hline
		\ddref{DD:Amplitude}
		& & & & &X& & &X& \\\hline
		\tref{T:Transformation}
		&X& & & &X&X& &X&X\\\hline
		\tref{T:Addition}
		& & &X& &X& & &X& \\\hline
		\tref{T:Subtraction}
		& & &X& &X& & &X& \\\hline
		\tref{T:Multiplication}
		& & &X& &X& & &X& \\\hline
		\tref{T:Division}
		& & &X& &X&X& &X&X\\\hline
		\tref{T:Amplitude}
		& & & & &X& & &X& \\\hline
		\iref{IM:CFScoeff}
		&X& & & &X&X& &X&X\\\hline
		\iref{IM:Addition}
		& & &X& &X& & &X& \\\hline
		\iref{IM:Subtraction}
		& & &X& &X& & &X& \\\hline
		\iref{IM:Multiplication}
		& & &X& &X& & &X& \\\hline
		\iref{IM:Division}
		& & &X& &X&X& &X&X\\\hline
		\iref{IM:Function}
		& &X& & &X& & &X& \\\hline
		\iref{IM:Amplitude}
		& & & & &X& & &X& \\\hline
		\iref{IM:ToleratedEquality}
		& & &X& &X& & &X& \\\hline
		\iref{IM:ConvertTo}
		& & & &X&X& &X&X& \\\hline
		\iref{IM:ConvertFrom}
		& & & &X&X& &X&X& \\\hline		
	\end{tabular}
	\caption{The traceability matrix between goals, 
	theoretical models, data definitions, and instances models 
	with the assumptions, calculations, and outputs}
	\label{Table:Traceability}
\end{table} 
\newpage

\bibliographystyle {plainnat}
\bibliography {../../refs/References}

\newpage
\appendix

\section{Theory for operations on CFS's}\label{Appendix:Operations}
In this appendix, we hereby give the proof of 
\tref{T:Addition}, \tref{T:Subtraction}, \tref{T:Multiplication}, 
and \tref{T:Amplitude} in corresponding paragraphs. The proof of 
\tref{T:Division} comes directly from the equation 
$[f(t)/g(t)]*g(t)=f(t)$.

In the following proofs, suppose we have two functions, 
$f(t)$ and $g(t)$ with existing IFS and CFS. The $n$ and $\omega$ of 
these IFS's and CFS's are the same, but with different $A_i$'s and $B_i$'s 
(denoted with $A_{f, i}$'s, $B_{f, i}$'s and $A_{g,i}$'s, 
$B_{g,i}$'s respectively).
From the definition of IFS, \ddref{DD:IFS}, we know that 
\begin{equation}\label{Eq:fDef}
	f(t)=\sum_{i=0}^{+\infty}A_{f, i}\cos(i\omega t)
	+\sum_{i=1}^{+\infty}B_{f, i}\sin(i\omega t),
\end{equation} and \begin{equation}\label{Eq:gDef}
	g(t)=\sum_{i=0}^{+\infty}A_{g, i}\cos(i\omega t)
	+\sum_{i=1}^{+\infty}B_{g, i}\sin(i\omega t).
\end{equation}  
\paragraph{Addition and Subtraction}\label{App-Para:Addition&Subtraction}
Like $f(t)$ and $g(t)$, we also know that 
\begin{equation}\label{Eq:f+gDef}
	f(t)+g(t)=\sum_{i=0}^{+\infty}A_{f+g, i}
	\cos(i\omega t)+\sum_{i=1}^{+\infty}B_{f+g, i}\sin(i\omega t).
\end{equation} 
By replacing $f(t)$ and $g(t)$ in \autoref{Eq:f+gDef} 
with \autoref{Eq:fDef} and \autoref{Eq:gDef}, we have
\begin{equation}\label{Eq:f+gCoeff}
	\begin{aligned}
	&\sum_{i=0}^{+\infty}A_{f+g, i}\cos(i\omega t)
	+\sum_{i=1}^{+\infty}B_{f+g, i}\sin(i\omega t)\\
	=&\sum_{i=0}^{+\infty}A_{f, i}\cos(i\omega t)+\sum_{i=1}^{+\infty}
	B_{f, i}\sin(i\omega t)+\sum_{i=0}^{+\infty}A_{g, i}\cos(i\omega t)
	+\sum_{i=1}^{+\infty}B_{g, i}\sin(i\omega t)\\
	=&\sum_{i=0}^{+\infty}(A_{f, i}+A_{g,i})\cos(i\omega t)
	+\sum_{i=1}^{+\infty}(B_{f, i}+B_{g, i})\sin(i\omega t)\\
	\end{aligned}
\end{equation}
By comparing the coefficients in \autoref{Eq:f+gCoeff}, we have
\begin{equation}\label{Eq:f+gConclusion}
	\begin{aligned}
	A_{f+g, i}&=A_{f,i}+A_{g, i}\\
	B_{f+g, i}&=B_{f,i}+B_{g, i}\\
	\end{aligned}
\end{equation}
for the IFS and CFS of $f(t)+g(t)$.

Likewise, we have similar conclusions for those of $f(t)-g(t)$.
\paragraph{Multiplication}\label{App-Para:Multiplication}
From \autoref{Eq:fDef} and \autoref{Eq:gDef}, we have
\begin{equation}
	f(t)*g(t)=[\sum_{i=0}^{+\infty}A_{f, i}\cos(i\omega t)
	+\sum_{i=1}^{+\infty}B_{f, i}\sin(i\omega t)]
	*[\sum_{i=0}^{+\infty}A_{g, i}\cos(i\omega t)
	+\sum_{i=1}^{+\infty}B_{g, i}\sin(i\omega t)]\\
\end{equation}
which is the summation the following 3 terms
\begin{equation}
	\begin{aligned}
	\textit{Term A:} &\sum_{i=0, j=0}^{+\infty}A_{f,i}A_{g,j}
	\cos(i\omega t)\cos(j\omega t)\\
	=&\frac{1}{2}\sum_{i=0, j=0}^{+\infty}A_{f, i}A_{g, j}\cos[(i+j)\omega t] 
	+ \frac{1}{2}\sum_{i=0, j=0}^{+\infty}A_{f, i}A_{g, j}\cos[(j-i)\omega t]\\
	=&\frac{1}{2}\sum_{i=0}^{+\infty}[\sum_{j=0}^{i}A_{f,j}A_{g,i-j}]
	\cos(i\omega t)+\frac{1}{2}\sum_{j=0}^{+\infty}A_{f,i}A_{g,i}
	+\frac{1}{2}\sum_{i=1}^{+\infty}[\sum_{j=0}^{+\infty}A_{f, j}A_{g, j+i}
	+\sum_{j=0}^{+\infty}A_{f, j+i}A_{g, j}]\cos(i\omega t)\\
	=&A_{f,0}A_{g,0}+\frac{1}{2}\sum_{j=0}^{+\infty}A_{f,}A_{g,j}+\frac{1}{2}
	\sum_{i=1}^{+\infty}[\sum_{j=0}^{i}A_{f,j}+A_{g,i-j}+\sum_{j=0}^{+\infty}
	(A_{f,i}A_{g,j+i}+A_{f,j+i}A_{g,i})]\cos(i\omega t)\\
	\textit{Term B:} &\sum_{i=0,j=1}^{+\infty}[A_{f,i}B_{g,j}\cos(i\omega t)
	\sin(j\omega t)+B_{g,i}A_{f,j}\sin(i\omega t)\cos(j\omega t)]\\
	=&\frac{1}{2}\sum_{i=0,j=1}^{+\infty}[A_{f,i}B_{g,j}\sin((i+j)\omega t)
	-A_{f,i}B_{g,j}\sin((i-j)\omega t)]\\+&\frac{1}{2}
	\sum_{i=0,j=1}^{+\infty}[B_{g,i}A_{f,j}\sin((i+j)\omega t)+B_{g,i}A_{f,j}
	\sin((i-j)\omega t)]\\
	=&\frac{1}{2}\sum_{i=0,j=1}^{+\infty}[A_{f,i}B_{g,j}+B_{g,i}A_{f,j}]
	\sin((i+j)\omega t)+\frac{1}{2}\sum_{i=0,j=1}^{+\infty}
	[B_{g,i}A_{f,j}-A_{f,i}B_{g,j}]\sin((i-j)\omega t)\\
	=&\frac{1}{2}\sum_{i=1}^{+\infty}[\sum_{j=1}^{+\infty}
	A_{f,i-j}B_{g,j}+B_{g,i-j}A_{f,j}]\sin(i\omega t)
	-\frac{1}{2}\sum_{i=1}^{+\infty}[B_{g,0}A_{f,i}-A_{f,0}B_{g,i}]
	\sin(i\omega t)\\
	=&\frac{1}{2}\sum_{i=1}^{+\infty}[\sum_{j=1}^{+\infty}(A_{f,i-j}B_{g,j}
	+B_{g,i-j}A_{f,j})-B_{g,0}A_{f,i}+A_{f,0}B_{g,i}]\sin(i\omega t)\\
	\textit{Term C:} &\sum_{i=1,j=1}^{+\infty}B_{f,i}B_{g,j}\sin(i\omega t)
	\sin(j\omega t)\\
	=&\frac{1}{2}\sum_{i=1,j=1}^{+\infty}B_{f,i}B_{g,j}[\cos((i-j)\omega t)
	-\cos((i+j)\omega t)]\\
	=&\frac{1}{2}[\sum_{j=1}^{+\infty}B_{f,j}B_{g,j}]\cos(0\omega t)
	+\frac{1}{2}\sum_{i=1}^{+\infty}[\sum_{j=1}^{+\infty}
	B_{f,i+j}B_{g,i}+B_{f,i}B_{g,i+j}]\cos(i\omega t)\\
	-&\frac{1}{2}\sum_{i=2}^{+\infty}[\sum_{j=1}^{i-1}B_{f,i-j}B_{g,j}]
	\cos(i\omega t)
	\end{aligned}
\end{equation}
We gather the coefficients of $\cos(i\omega t), i=0:n$ 
and $\sin(i\omega t), i=1:n$ respectively, 
remove any terms containing 
$A_{f,k}$, $B_{f,k}$, $A_{g,k}$, 
and $B_{g,k}$ for any $k\geq n$ 
($k$ being either $i$, $j$, $i-j$, $i+j$, or $j-1$) 
as the result of a cut-off, 
and get the equations in \tref{T:Multiplication}.
\paragraph{Amplitude}\label{App-Para:Amplitude}
Quoted from \ddref{DD:Amplitude}, we have
\begin{equation}\label{Eq:Amp}
	\mathit{Amp}(\mathit{CFSf})=\sqrt{\frac{\omega}{2\pi}
	\int_{-\pi/\omega}^{\pi/\omega}\mathit{App}^2(\mathit{CFSf}, t)\text{d} t}
\end{equation}
In \autoref{Eq:Amp}, replacing $\mathit{App}
(\mathit{CFSf}, t)$ with its definition in \ddref{DD:Approximation}, 
we have
\begin{equation}\label{Eq:Amp1}
	\mathit{Amp}(\mathit{CFSf})=\sqrt{\frac{\omega}{2\pi}
	\int_{-\pi/\omega}^{\pi/\omega}[\sum_{i=0}^{n}A_{f,i}\cos(i\omega t)
	+\sum_{i=1}^{n}B_{f,i}\sin(j\omega t)][\sum_{j=0}^{n}A_{f,j}
	\cos(j\omega t)+\sum_{j=1}^{n}B_{f,j}\sin(j\omega t)]\text{d} t}
\end{equation}
The part inside the integration in \autoref{Eq:Amp1} can be expressed as
the summation of the following three expressions
\begin{equation}\label{Eq:AmpTerms}
\begin{aligned}
&\int_{-\pi/\omega}^{\pi/\omega}\sum_{i=0,j=0}^{n}A_{f,i}A_{f,j}
\cos(i\omega t)\cos(j\omega t)\text{d} t\\
&\int_{-\pi/\omega}^{\pi/\omega}\sum_{i=0,j=1}^{n}A_{f,i}B_{f,j}
\sin(i\omega t)\cos(j\omega t)\text{d} t\\
&\int_{-\pi/\omega}^{\pi/\omega}\sum_{i=1,j=0}^{n}B_{f,i}A_{f,j}
\sin(i\omega t)\cos(j\omega t)\text{d} t\\
&\int_{-\pi/\omega}^{\pi/\omega}\sum_{i=1,j=1}^{n}B_{f,i}B_{f,j}
\sin(i\omega t)\sin(j\omega t)\text{d} t\\
\end{aligned}
\end{equation}
The integration part of these expressions are
$\int_{-\pi/\omega}^{\pi/\omega}\cos(i\omega t)
\cos(j\omega t)$, $\int_{-\pi/\omega}^{\pi/\omega}\cos(i\omega t)
\sin(j\omega t)$, and $\int_{-\pi/\omega}^{\pi/\omega}\cos(i\omega t)
\sin(j\omega t)$. 
Calculation shows that the integration results are 
\begin{equation}\label{Eq:coscos}
\int_{-\pi/\omega}^{\pi/\omega}\cos(i\omega t)\cos(j\omega t)=\begin{cases}
2\pi/\omega, &i=j=0;\\
\pi/\omega, &i=j\neq 0;\\
0, &i\neq j.
\end{cases}
\end{equation} 
\begin{equation}\label{Eq:cossin}
	\int_{-\pi/\omega}^{\pi/\omega}\cos(i\omega t)\sin(j\omega t)=0
\end{equation}
and
\begin{equation}\label{Eq:sinsin}
	\int_{-\pi/\omega}^{\pi/\omega}\cos(i\omega t)\sin(j\omega t)=\begin{cases}
	\pi/\omega, &i=j\\
	0, &i\neq j\\
	\end{cases}
\end{equation}
Replacing integration in \autoref{Eq:AmpTerms} 
with integration results \autoref{Eq:coscos}, 
\autoref{Eq:cossin}, and \autoref{Eq:sinsin}, and we have the following results.
\begin{equation}\label{Eq:AmpTermsResult}
\begin{aligned}
&\int_{-\pi/\omega}^{\pi/\omega}\sum_{i=0,j=0}^{n}A_{f,i}A_{f,j}
\cos(i\omega t)\cos(j\omega t)\text{d} t
=\frac{2\pi}{\omega}A_0^2+\frac{\pi}{\omega}\sum_{i=1}^{n}A_i^2\\
&\int_{-\pi/\omega}^{\pi/\omega}\sum_{i=0,j=1}^{n}A_{f,i}B_{f,j}
\sin(i\omega t)\cos(j\omega t)\text{d} t
=0\\
&\int_{-\pi/\omega}^{\pi/\omega}\sum_{i=1,j=0}^{n}B_{f,i}A_{f,j}
\sin(i\omega t)\cos(j\omega t)\text{d} t
=0\\
&\int_{-\pi/\omega}^{\pi/\omega}\sum_{i=1,j=1}^{n}B_{f,i}B_{f,j}
\sin(i\omega t)\sin(j\omega t)\text{d} t
=\frac{\pi}{\omega}\sum_{i=1}^{n}B_i^2\\
\end{aligned}
\end{equation}

Summing them up and putting the result back into 
\autoref{Eq:Amp1}, we have the expression 
in \tref{T:Amplitude}.

\end{document}