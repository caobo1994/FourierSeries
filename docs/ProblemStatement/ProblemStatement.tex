\documentclass{article}

\usepackage{tabularx}
\usepackage{booktabs}

\usepackage{amsfonts}

\title{CAS 741: Problem Statement\\A Fourier Series Library}

\author{Bo Cao, caob13@mcmaster.ca}

\date{\today}

\input{../Comments}

\begin{document}

\maketitle

\section{Goal}
I intend to implement a library for Fourier series related computation. I will mainly implement the following functions.
\begin{itemize}
	\item Compute the Fourier series of a given function.
	\item Compute the value of a function with given variable value and the Fourier series of this function.
	\item Implement the addition, subtraction, multiplication and division of Fourier series. That is, suppose that the functions $f(t)$ and function $g(t)$ have Fourier series $F$ and $G$, respectively, this library computes the Fourier series of $f(t)+g(t)$, $f(t)-g(t)$, $f(t)*g(t)$, and $f(t)/g(t)$ from $F$ and $G$. 
	\item Implement some basic functions (sin, exp, etc.) of Fourier series. That is, suppose a function $f(t)$ with known Fourier series $F$, and a known basic function $g()$, we would like to compute the Fourier series of $g(f(t))$ from $F$. 
	\item Formatted input and output of Fourier series. 
\end{itemize}

The aforementioned function shall be a $\mathbb{R}\rightarrow\mathbb{R}$ function, whose Fourier series exists. Instead of verification by this library (due to foreseeable technical difficulties), this property shall be guaranteed by the users. 

\section{Motivation}
An equation solver, whose development I currently participate in, utilizes Taylor series, and I intend to see if its Taylor series parts can be replaced by Fourier series ones. For this purpose, I need to implement a basic library for the Fourier series related computations, hence this library.

\section{Environment}
This library is developed and tested on the 64-bit Ubuntu 18.04, and might be tested on the macOS 10.14 and 10.15 (after its release). 

This library's compatibility with other environments are not guaranteed nor tested, although I guess that it should be compatible with 64-bit Ubuntu 16.04.
  
%\begin{table}[hp]
%\caption{Revision History} \label{TblRevisionHistory}
%\begin{tabularx}{\textwidth}{llX}
%\toprule
%\textbf{Date} & \textbf{Developer(s)} & \textbf{Change}\\
%\midrule
%Date1 & Name(s) & Description of changes\\
%Date2 & Name(s) & Description of changes\\
%... & ... & ...\\
%\bottomrule
%\end{tabularx}
%\end{table}
%
%Put your problem statement here.  Comments to you can be added, like this:
%
%\wss{comment}
%
%You can also leave comments for yourself, like this:
%
%\an{comment}

\end{document}