\documentclass{article}

\usepackage{tabularx}
\usepackage{booktabs}

\usepackage{amsfonts}

\title{CAS 741: Problem Statement\\A Fourier Series library}

\author{Bo Cao, caob13@mcmaster.ca}

\date{\today}

%% Comments

\usepackage{color}

\newif\ifcomments\commentstrue

\ifcomments
\newcommand{\authornote}[3]{\textcolor{#1}{[#3 ---#2]}}
\newcommand{\todo}[1]{\textcolor{red}{[TODO: #1]}}
\else
\newcommand{\authornote}[3]{}
\newcommand{\todo}[1]{}
\fi

\newcommand{\wss}[1]{\authornote{blue}{SS}{#1}} 
\newcommand{\plt}[1]{\authornote{magenta}{TPLT}{#1}} %For explanation of the template
\newcommand{\an}[1]{\authornote{cyan}{Author}{#1}}


\begin{document}

\maketitle

\section{Goal}
We intend to implement a library for Fourier Series related computation. We mainly implement the following functions.
\begin{itemize}
	\item Compute the Fourier series of a given function.
	\item Compute the value of a function with given variable value and the Fourier Series of a function.
	\item Implement the addition, subtraction, multiplication and division of Fourier Series.
	\item Implement some basic functions (sin, exp, etc.) of Fourier Series.
	\item Input and output of Fourier Series. 
\end{itemize}

The aforementioned function shall be a function $\mathbb{R}\rightarrow\mathbb{R}$, whose Fourier Series exists.

\section{Motivation}
An equation solver, whose development I currently participate in, utilizes Taylor Series, and I intend to see if the Taylor Series in it can be replaced by Fourier Series. For this purpose, I need to implement a basic library for the Fourier Series related computations, hence this library.

\section{Environment}
This library is developed and tested on a fully updated 64-bit Ubuntu 18.04, and might be tested on the macOS 10.14 and 10.15 (after its release). 

This library's compatibility with other environments are not guaranteed nor tested, although I guess that it should be compatible with 64-bit Ubuntu 16.04.
  
%\begin{table}[hp]
%\caption{Revision History} \label{TblRevisionHistory}
%\begin{tabularx}{\textwidth}{llX}
%\toprule
%\textbf{Date} & \textbf{Developer(s)} & \textbf{Change}\\
%\midrule
%Date1 & Name(s) & Description of changes\\
%Date2 & Name(s) & Description of changes\\
%... & ... & ...\\
%\bottomrule
%\end{tabularx}
%\end{table}
%
%Put your problem statement here.  Comments to you can be added, like this:
%
%\wss{comment}
%
%You can also leave comments for yourself, like this:
%
%\an{comment}

\end{document}