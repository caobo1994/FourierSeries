\documentclass{beamer}
\title{CA presentation\\A Fourier Series Library}
\author{Bo Cao, caob13@mcmaster.ca}
\date{\today}
\begin{document}
	\begin{frame}
		\maketitle
	\end{frame}
\begin{frame}[allowframebreaks]
\frametitle{Table of Contents}
\tableofcontents[sections={1-3}]
\framebreak
\tableofcontents[sections={4-7}]
\end{frame}

	\section{Introduction}
	\subsection{Purpose of Document}
	\begin{frame}
		\frametitle{Purpose of Document}
		An overview of a library for Fourier Series related computation.
		\begin{itemize}
			\item Introduce underlying mathematical theory.
			\item Introduce necessary adaptations of theory.
			\item Define a data structure storing Fourier Series data.
			\item Design and implement operations. (implementation not covered in the following slides)
			\item Review verification of library. (only a simple introduction shown)
		\end{itemize}
	\end{frame}

	\subsection{Scope of the Family}
	\begin{frame}
		\frametitle{Scope of the Family}
		\begin{itemize}
			\item Limited length of Fourier Series - limited accuracy.
			\item Only accepts functions able to be converted to Fourier Series, no checking - hard to implement, leave to users.
			\item Binary operations require frequency agreements - better with explicit frequency conversion.
			\item Supports popular floating point types - but requires basic arithmetic operations and input/output.  
		\end{itemize}
	\end{frame}

	\subsection{Characteristics of Intended Reader}
	\begin{frame}
		\frametitle{Characteristics of Intended Reader}
		\begin{itemize}
			\item Developers, maintainers and contributors.
			\item Potential users who want deep knowledge of this library (for compatibility analysis, code audition, etc.)
		\end{itemize}
	\end{frame}

	\subsection{Organization of Document}
	\begin{frame}
		\frametitle{Organization of Document}
		Same as the template.
	\end{frame}

	\section{General System Description}
	\subsection{Potential System Contexts}
	\begin{frame}
		\frametitle{Potential System Contexts}
		User Responsibilities:
		\begin{itemize}
			\item Source code distribution, compatibility between compilers - compile by users.
			\item Check of input legality - library can detect some, but not all. Users have better understanding of library's used environments.
		\end{itemize}
		Library responsibilities:
		\begin{itemize}
			\item Report found input illegality - wrong format, etc.
			\item Perform operations in promised way.
			\item Get results within promised accuracy, analyze error when possible.
			\item Output within given format.
			\item Crash only when it is a must - zero memory leakage, no excessive CPU usage, etc.
		\end{itemize}
	\end{frame}

	\subsection{Potential User Characteristics}
	\begin{frame}
		\frametitle{Potential User Characteristics}
		Who uses this library?
		\begin{itemize}
			\item Scientific/industrial software developers - almost everyone have degrees/abundant knowledge in coding and math.
			\item Students preparing for these jobs - finished basic courses.
		\end{itemize}
		What do they know?
		\begin{itemize}
			\item Basic knowledge in Fourier Series - covered in advanced calculus courses, even real analysis
			\item Basic analysis of error estimation - taught in introductory scientific computation courses. (CAS 4X03, CAS 708/ CSE 700 as examples)
			\item Understanding of one of the languages in which this library provide APIs (C/C++, maybe Python, MATLAB, etc.) 		
		\end{itemize}
		These are the characteristics of potential users.
	\end{frame}

	\subsection{Potential System Constraints}
	\begin{frame}
		\frametitle{Potential System Constraints}
		\begin{itemize}
			\item None until now.
			\item Possible suggestions welcomed.
		\end{itemize}
	\end{frame}

	\section{Commonalities}
	\subsection{Background Overview}
	\begin{frame}
		\frametitle{Overview of Related Libraries}
			A list of related libraries
		\begin{itemize}
			\item FFTW: \url{www.fftw.org};
			\item Java Fourier Transform Library: \url{https://github.com/tambapps/fourier-transform-library}, written in Java;
			\item FINUFFT: \url{https://finufft.readthedocs.io/en/latest/};
			\item Fourier Transform in MATLAB:
			\url{https://www.mathworks.com/help/symbolic/fourier.html}, possible operations in other packages.
		\end{itemize}
		Analysis:
		\begin{itemize}
			\item Most written in C/C++;
			\item Most only implement transformation;
			\item Fast transformation.
		\end{itemize}

	\end{frame}

	\begin{frame}
	\frametitle{What do we focus on?}
		Requirements on our library:
\begin{itemize}
	\item Focus on operation;
	\item Might need to implement conversion to/from them.
\end{itemize}
	\end{frame}

	\subsection{Terminology and Definitions}
	\begin{frame}
		\frametitle{Terminology and Definitions}
		Fourier Series:
		\begin{equation}
			f(t) = A_0+\sum_{i=1}^{+\infty}A_i\cos(i\omega t)+\sum_{i=1}^{+\infty}B_i\sin(i\omega t).
		\end{equation}
		\begin{itemize}
			\item $f(t)$: transformed function;
			\item $\omega$: base frequency;
			\item $A_i, B_i$: Fourier Series.
		\end{itemize}
	\end{frame}

	\subsection{Data Definitions}
	\begin{frame}
	\frametitle{Data Definitions}
	What is enough to describe a Fourier Series?
\begin{itemize}
	\item $\omega$;
	\item $A_i, B_i$.
\end{itemize}
Can our library handle infinite series? No.

What to do? Cut off!

Where? Let users decide.

Additional information: cut-off position $n$, only $A_i, i=0:n$ and $B_i, i=1:n$ are computed, stored and outputted.

Will this library estimate error? Maybe. See if it is easy to implement.
	\end{frame}

	\subsection{Goal Statements}
	\begin{frame}
		\frametitle{Goal Statements: Quoted from Problem Statements}
			I intend to implement a library for Fourier series related computation. I will mainly implement the following functions.
		\begin{itemize}
			\item Compute the Fourier series of a given function.
			\item Compute the value of a function with given variable value and the Fourier series of this function.
			\item Implement the addition, subtraction, multiplication and division of Fourier series. That is, suppose that the functions $f(t)$ and function $g(t)$ have Fourier series $F$ and $G$, respectively, this library computes the Fourier series of $f(t)+g(t)$, $f(t)-g(t)$, $f(t)*g(t)$, and $f(t)/g(t)$ from $F$ and $G$. 
			\item Implement some basic functions (sin, exp, etc.) of Fourier series. That is, suppose a function $f(t)$ with known Fourier series $F$, and a known basic function $g()$, we would like to compute the Fourier series of $g(f(t))$ from $F$. 
			\item Formatted input and output of Fourier series. 
		\end{itemize}
	\end{frame}

	\begin{frame}
		\frametitle{Possible Extensions}
		\begin{itemize}
			\item Comparison between Fourier series;
			\item Amplitude of Fourier Series;
			\item Base frequency reduction ($\omega \rightarrow(1/k)\omega$, $k\in\mathbb{Z}^{+}$); 
			\item Conversion from/to other data formats.
	\end{itemize}
	\end{frame}

	\subsection{Theoretical Models}
	\begin{frame}
		\frametitle{Theoretical Models}
		\begin{itemize}
			\item Conversion from functions: Fast Fourier Transform (FFT).
			\item Computation of function values: direct and easy.
			\item Addition, subtraction, multiplication: use the operations of two Fourier Series to find relationships between coefficients.
			\item Division: a little hard, covered later.
			\item Functions of Fourier Series: convert functions to Taylor Series, replace independent variables with Fourier Series, find relationships between coefficients.
			\item Input/output: direct and easy.
		\end{itemize}
	\end{frame}

	\begin{frame}
		\frametitle{How to define and compute division?}
		Two ways for computing the Fourier Series of $f(t)/g(t)$ from $f(t)$'s Fourier Series $A_{f,i}, B_{f,i}$ and Fourier Series $A_{g,i}, B_{g,i}$.
		
		Define the solution by $A_{d,i}, B_{d,i}$.
		
		Ways:
		\begin{enumerate}[W{a}y 1.]
		\item Use $[f(t)/g(t)] * g(t) = f(t)$, build linear equations among $A_{f,i}, B_{f,i}$, $A_{g,i}, B_{g,i}$, $A_{d,i}, B_{d,i}$. Solve equations for $A_{d,i}, B_{d,i}$.

		\item Calculate Fourier Series of $1/g(t)$ like Way 1, and multiply it with that of $f(t)$.
		\end{enumerate}
		Same? Not with cut-off errors, but can be proven equal within this error.
		
		Both throws exceptions when unique solution cannot be found for linear equations. (Same as division by zero in normal divisions?)
		
		Advantage compared to others.
		\begin{enumerate}[W{a}y 1.]
			\item Intuitive.
			\item Reusability for multiple $f(t)$'s divided by one $g(t)$.
		\end{enumerate}
		Have not decided which one to use. Any suggestions?
		
	\end{frame}

	\section{Variabilities}
		\subsection{Assumptions}
		\begin{frame}
		\frametitle{Variability in Assumptions}
		\begin{itemize}
			\item How long are the series before cut-off?
			\item Data type to store coefficients and base frequency.
			\item How to get the input? Conversion from what other data formats?
		\end{itemize}
		\end{frame}
	
		\subsection{Calculation}
		\begin{frame}
			\frametitle{Variability in Calculation}
			\begin{itemize}
				\item Fourier transform part: one algorithm, multiple ones chosen by user, or chosen by library?
				\item What linear solver to use?
				\item Cut-off method: just cut-off or let cut-off terms slightly modify remaining terms?
				\item Cut-off in Taylor Series. How long? Relationship with Fourier Series cut-off? Who makes the decision?
			\end{itemize}
		\end{frame}
	
		\begin{frame}
			\frametitle{Variability in Output}
			\begin{itemize}
				\item Output/conversion format. Which ones to implement?
				\item Accuracy in formatted output.
				\item Too large to be outputted? Maybe not.
			\end{itemize}
		\end{frame}
	
	\section{Requirements}
	\subsection{Functional Requirements}
	\begin{frame}
	\frametitle{Functional Requirements on Functions Converted to Fourier Series}
	Quoted from the Problem Statement:
	
	The aforementioned function shall be a $\mathbb{R}\rightarrow\mathbb{R}$ function, whose Fourier series exists. Instead of verification by this library (due to foreseeable technical difficulties), this property shall be guaranteed by the users. 
	\end{frame}
	
	
	\begin{frame}
		\frametitle{Functional Requirements: Others}
		\begin{itemize}
			\item Size: not too large. (usually)
			\item Overflow? Users shall prevent, library just throws exception.
		\end{itemize}
	\end{frame}
	
	\begin{frame}
		\frametitle{Non-functional requirements}
		\begin{itemize}
			\item Implement easy algorithms first, no much need on speed - low data size.
			\item Compiler/dependent library compatibility - copy/implement solvers or call them?
		\end{itemize}
	\end{frame}
	
	\section{Verification}
	\begin{frame}
		\frametitle{Verification}
		\begin{itemize}
			\item Verify with simple functions.
			\item Error estimation needed for verification? Maybe implement a simple but relaxed error estimation.
			\item How many kinds of input? All $f(t)$ and $g(t)$ cover as many types of functions as possible.
		\end{itemize}
		To be expanded in the future.
	\end{frame}
	
	\section{Traceability}
	\begin{frame}
		\frametitle{Traceability}
		TBD.
	\end{frame}

\end{document}